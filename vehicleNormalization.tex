\documentclass[12pt]{article}   % others are available, like book

\usepackage{amssymb}            % adds more math symbols to LaTeX's basic set
\usepackage{graphicx}           % for importing/adjusting images
\usepackage{pifont}
\usepackage{graphicx}
\usepackage{mathtools}
\usepackage{blkarray, bigstrut}

\newcommand{\xmark}{\ding{55}}
\newcommand\floor[1]{\lfloor#1\rfloor}
\newcommand\ceil[1]{\lceil#1\rceil}

\setlength{\topmargin}{-0.50in}     % in == inches (others:  cm, mm, pt, ...)
\setlength{\textheight}{9.25in}     % what's left-over is the bottom margin
\setlength{\textwidth}{6.625in}
\setlength{\oddsidemargin}{0.0in}   % right-side pages in a magazine
\setlength{\evensidemargin}{0.0in}  % left-side

\setlength{\parindent}{0.0cm}	    % don't indent first lines of paragraphs
\setlength{\parskip}{0.4cm}   

\title{vehicle3NF}


\begin{document}

\underline{Steps to get to 3NF}
\begin{enumerate}
    \item Create initial relational design
    \item Identify the FDs
    \item Construct decomposed schema for which:
          \begin{itemize}
              \item Natural joins do not add spurious tuples
              \item All relations are in at least 3NF
              \item All FDs are retained or can be reconstructed
          \end{itemize}
\end{enumerate}

\underline{Step \#1 (Create initial relational design)}\\
\\
Vehicle (DocumentID(\textbf{FK}), LicenseNumber, Make, Model, VIN(\textbf{PK}))\\

\underline{Step \#2 (Identify the FDs)}\\
\\
\underline{Notes:}\\
I know that for a given CK it should determine all attributes in the relation, but is that true in this case? There can be multiple Documents associated with one Vehicle meaning, multiple DocumentIDs associated with a Vehicle.\\

\{VIN\} $\longrightarrow$ \{Make, Model, LicenseNumber\}\\
\{LicenseNumber, Make, Model\} $\longrightarrow$ \{VIN\}\\
\{VIN\} $\longrightarrow$ \{LicenseNumber\}\\
\{VIN\} $\longrightarrow$ \{Make\}\\
\{VIN\} $\longrightarrow$ \{Model\}\\
\{VIN\} $\longrightarrow$ \{VIN\}\\

\underline{Step \#3 (1NF)}\\
\\
None of the attributes in Vehicle are set-valued $\therefore$ Vehicle is in 1NF.\\

\underline{Step \#3 (2NF)}\\
\\
The only FD that could cause the relation to not be in 2NF is: \{LicenseNumber, Make, Model\} $\longrightarrow$ \{VIN\}, it is in the form X $\longrightarrow$ Y and X is composite. However Y is a prime attribute, so we can ignore this FD as well. $\therefore$ Vehicle is in 2NF.\\
\newpage

\underline{Step \#3 (3NF)}\\
\\
With the FD: \{LicenseNumber, Make, Model\} $\longrightarrow$ \{VIN\}, \{VIN\} is a prime attribute, condition (b) is satisfied.\\

For all the other FDs condition (a) is satisfied. $\therefore$ Vehicle is in 3NF.



\end{document}

